\documentclass[12pt]{article}

\usepackage{graphicx}
\graphicspath{{Images/}}


\title{\Large NATIONAL INSTITUTE OF TECHNOLOGY RAIPUR}
\author{\large Submitted by: Harshita Upendra Sakhare\\Roll.No:21111049\\Submitted to: Prof. Saurabh Gupta\\5 BIOMEDICAL DEVICES}




\begin{document}

\begin{figure}
\centering
\includegraphics[scale=0.3]{NITRR.jpg}
\end{figure}

\maketitle
\tableofcontents
\clearpage








\section{INTRODUCTION}
{\large A medical device can be any instrument, apparatus, implement, machine, appliance, implant, reagent for in vitro use, software, material or other similar or related article, intended by the manufacturer to be used, alone or in combination for a medical purpose.\\


5 Devices are:\\
1. ECG Electrocardiogram\\2. Dental X-ray\\3. Chest X-ray\\4. Pacemaker\\5. Camera pill}



\section{ELECTROCARDIOGRAM (ECG)}




\subsection{WHAT IS ELECTROCARDIOGRAM (ECG)?}


\begin{figure}[h]
\centering
\includegraphics[scale=0.6]{EKG Machine.jpeg}
\caption{Electrocardiogram (ECG or EKG) Machine}
\end{figure}



{\large Electrocardiography is the process of producing an electrocardiogram(ECG is also known as EKG) a recording of the heart,s electrical activity. It is an electrogram of the heart which is graph of voltage versus time of the electrical activity of the heart using electrodes placed on the skin.}


\subsection{MEDICAL USES}


{\large These are some medical uses of ECG:

\begin{itemize}
\item Symptoms such as shortness of breath, murmurs, fainting.
\item Myocardial Infraction(Heart attack)
\item Medication Monitoring and management of overdose
\item Electrolyte abnormalities such as hyperkalemia.
\end{itemize}
 There are many more uses of ECG.}


\subsection{ELECTROCARDIOGRAPH MACHINE OR ECG MACHINES}





{\large ECG are recorded by machines that consist of a set of electrodes connected to a central unit. Early ECG machines were constructed with analog electronics, where the signal drove a motor to print out the signal onto paper. Today, ECG use analog to digital converters to convert the electrical activity of the heart to a digital signal.}



\subsection{ELECTRODES AND LEADS}




{\large Electrodes are the actual conductive pads attached to the body surface.Any pair of electrodes can measure the electrical potential difference between the two corresponding locations of attachment.Such a pair forms a lead.However, "leads" can also be formed between a physical electrode and a virtual electrode, known as Wilson's central terminal (WCT).}


\subsection{ELECTROPHYSIOLOGY}

{\large The study of the conduction system of the heart is called cardiac electrophysiology (EP).An EP study is performed via a right-sided cardiac catheterization:a wire with an electrode at its tip is inserted into the right heart chambers from a peripheral vein, and placed in various positions in close proximity to the conduction system so that the electrical activity of the system can be recorded.}





\subsection{INTERPRETATION}


{\large Interpretation of the ECG is fundamentally about understanding the electrical conduction system of the heart. Normal conduction starts and propagates in a predictable pattern, and deviation from this pattern can be a normal variation or be pathological. An ECG does not equate with mechanical pumping activity of the heart, for example,pulseless electrical activity produces an ECG but no pulses are felt.}





\subsection{DIAGNOSIS}


{\large Numerous diagnoses and findings can be made based upon ECG, and many are discussed above.\\The following is an organised list of possible ECG-based diagnoses:}


\begin{itemize}
{\large \item Atrial fibrillation and atrial flutter without rapid ventricular response
\item Premature atrial contraction (PACs)and premature ventricular contraction (PVCs)
\item Sinus arrhythmia
\item Sinus bradycardia and sinus tachycardia
\item Sinus pause and sinoatrial arrest}
\end{itemize}



\subsection{HISTORY OF ECG}


\begin{figure}[h]
\centering
\includegraphics[scale=1]{history of ecg.jpg}
\caption{Electrocardiogram of olden time}
\end{figure}



{\large The term 'electrocardiogram' used to describe these wave forms was first coined by Einthoven at the Dutch Medical Meeting of 1893(8,10).In 1901, he successfully developed a new string galvanometer with very high sensitivity, which he used in his electrocardiograph. \\ In 1924, Einthoven was awarded the Nobel prize in Medicine for his pioneering work in developing the ECG. By 1927, General Electric had developed a portable apparatus that could produce ECG without the use of the string galvanometer. In 1937, Taro Takemi invented a new portable ECG machine. In 1942, Emanuel Goldberger increases the voltage of Wilson's unipolar leads by 50 percent and creates the augmented limbs leads aVR, aVL and aVF. when added to Einthoven's three limb leads and the six chest leads we arrive at the 12- lead ECG that is used today.In the late 1940's Rune Elmqvist invented an inkjet printer with good frequency response and direct recording of ECG on paper.}







\section{PACEMAKER}




\subsection{WHAT IS PACEMAKER}



\begin{figure}[h]
\centering
\includegraphics[scale=0.3]{Pacemaker.2.jpg}
\caption{Pacemaker}
\end{figure}



{\large A cardiac pacemaker or artificial pacemaker, is a medical device that generates electrical impulses delivered by electrodes to cause the heart muscle chamber to contract and therefore pump blood by doing so this device replaces and/or regulates the function of the electrical conduction system of the heart.}





\subsection{HISTORY OF PACEMAKER}


\begin{figure}[h]
\centering
\includegraphics[scale=0.3]{old pacemaker.jpg}
\caption{Old Pacemaker}
\end{figure}


{\large Rune Elmqvist (1906-1996) developed the first implantable pacemaker in 1958, working under the direction of Ake Senning, senior physician and cardiac surgeon at the Karolinska University hospital in Solna Sweden. Rune Elmqvist initially worked as a medical doctor but later worked as an engineer and inventor.}




\subsection{METHOD OF PACING}


\begin{itemize}
{\large \item Percussive pacing
\item Transcutaneous pacing
\item Epicardial pacing (temporary)
\item Transvenous pacing (temporary)
\item Permanent transvenous pacing
\item Leadless pacing}
\end{itemize}


\subsection{BASIC FUNCTION}


{\large Modern pacemaker usually have multiple functions. The most basic form of monitors the heart's native electrical rhythm. When the pacemaker wire or "lead" does not detect heart electrical activity in the chamber atrium or ventricle within the normal beat to beat time period most commonly 1 second it will stimulate either the atrium or the ventricle with a short low voltage pulse. If it does not sense electrical activity, it will hold of stimulating. This sensing and stimulating activity continues on a beat by beat basis and is called demand pacing.}




\section{CHEST X-RAY}



\subsection{WHAT IS CHEST X-RAY?}



\begin{figure}[h]
\centering
\includegraphics[scale=0.2]{Normal Chest X-ray.jpg}
\caption{Normal X-ray}
\end{figure}



{\large The chest x-ray is the most commonly perform diagnostic x-ray examination. A chest x-ray produces images of the heart, lungs, airways, blood vessels and the bones of the spines and the chest.\\An x-ray exam helps doctors diagnose and treat medical condition it exposes you to small dose of ionizing radiation to produce pictures of the inside of the body. X-rays are the oldest and most often used form of medical imagining.}



\subsection{TYPES OF CHEST X-RAY}

{\large The most common views of Chest x-rays are:\\
\hspace{1cm}
1. Posteroanterior Chest X-ray: In posteroanterior(PA) view, the x-ray source is positioned so that the x-ray beam enters through the posterior (back) aspect of the chest and exists out of the anterior(front) aspect, were the beam is detected. To obtain this view, the patient stands facing a flat surface behind which is an x-ray detector.\\
\hspace{1cm}
2. Anteroposterior Chest X-ray: An x-ray picture in which the beams pass from front to back (anteroposterior). As opposed to a PA (Posteroanterior) film in which the rays pass through the body from back to front.\\
\hspace{1cm}
3. Lateral Chest X-ray: The lateral chest view examines the lungs, bony thoracic cavity, mediastinum, and great vessels.}


\subsection{PROCEDURE OF CHEST X-RAY}

{\large A chest x-ray is a safe and painless test that uses a small amount of radiation to take a picture of a persons chest. During the examination, an x-ray machine sends a beam of radiation through the chest, and an image is recorded on special film or a computer.}

\subsection{WHY IS CHEST X-RAY DONE?}

{\large If you go to your doctor or the emergency room with chest pain, a chest injury or shortness of breath, you will typically get a chest x-ray. The image helps your doctor determine whether you have heart problems, a collapsed lung, pneumonia, broken ribs, emphysema, cancer or any of several other conditions.}



\subsection{BENEFITS OF CHEST X-RAY}



{\large Chest x-rays can detect cancer, infection or air collecting in the space around a lung, which can cause the lung to collapse. They can also show chronic lung condition, such as emphysema or cystic fibrosis, as  well as complications related to these conditions. Heart related lung problems.}



\subsection{RISKS OF CHEST X-RAY}


{\large While x-rays are linked to a slightly increased risk of cancer, there is an extremely low risk of short term side effects. Exposure to high radiation levels can have a range of effects, such as vomiting, bleeding , fainting, hair loss, and the loss of skin and hair.}


\section{CAMERA PILL OR CAPSULE ENDOSCOPY}

\subsection{WHAT IS CAMERA PILL OR CAPSULE ENDOSCOPY?}


\begin{figure}[h]
\centering
\includegraphics[scale=0.3]{Camera pill.jpg}
\caption{Camera Pill}
\end{figure}


{\large During a capsule endoscopy, also known as capsule enteroscopy or small bowel endoscopy, a tiny camera captures images of your digestive tract. Your doctor may order a capsule endoscopy in order to view your small intestine since that area is hard to reach using other endoscopic procedures. You swallow a capsule, which moves through the digestive tract taking thousands of pictures that your doctor can examine.}


\subsection{PROCEDURE OF CAPSULE ENDOSCOPY}

{\large A typical capsule endoscopy procedure involves these steps:\\}
\begin{enumerate}
{\large \item When you visit your doctor, they will inform you about the procedure, along with what to expect.
\item You'll be ask to wear a recording device on your waist throughout the procedure. This device will record and store picture taken by the capsule camera as it moves through your gut. Some recording devices even have electrode patches that must be applied onto the skin of your chest or abdomen.
\item Your doctor will then ask you to swallow the pill size wireless video endoscopy capsule with some water.
\item Once you swallow the capsule, you can get back to daily life for the next 8 hours.
\item As the capsule journeys through your digestive system, your doctor will instruct you too wait atleast 2 hours before drinking clear liquids, wait atleast for 4 hours before having a snack, ensure that you keep the recording device safe and dry by avoiding bathing or swimming.}

\end{enumerate}




\subsection{WHY CAPSULE ENDOSCOPY DONE?}


{\large Capsule  endoscopy is used to diagnose problems in your digestive system. Capsule endoscopy can help your doctor:}

\begin{itemize}
{\large \item Diagnose gastrointestinal cancer 
\item Identify the cause of unexplained abdominal pain
\item Conduct follow up testing
\item Detect polyps or tumors in your digestive tract}
\end{itemize}


\subsection{BENEFITS OF CAPSULE ENDOSCOPY}


{\large Diagnose inflammatory bowel diseases, such as Crohn's disease. Capsule endoscopy can reveal areas of inflammation in the small intestine. Diagnose cancer. Capsule endoscopy can show tumors in the small intestine or other parts of the digestive tracts.}



\subsection{RISKS OF CAPSULE ENDOSCOPY}


{\large The primary risk with capsule endoscopy is possible retention of the device in the small bowel. In patients who undergo the test to evaluate for bleeding, the risk is very low, approximately one to two percent. For patient with crohn's disease, the risk may increase to 4 to 5 percent.}





\section{DENTAL X-RAY}

\subsection{WHAT IS DENTAL X-RAY?}

\begin{figure}[h]
\centering
\includegraphics[scale=0.5]{dental x-ray.jpg}
\end{figure}


{\large Dental x-rays are images of your teeth that your dentist uses to evaluate your oral health. These x-rays are used with low levels of radiation to capture images of the interior of your teeth and gums. This can help your dentist to identify problems, like cavities, tooth decay, and impacted teeth. Dental x-rays may seem complex, but they're actually very common tools that are just as important as your teeth cleanings.}


\subsection{TYPES OF DENTAL X-RAYS}

{\large There are several types of dental x-rays, which record slightly different views of your mouth. The most common are intraoral x-rays, such as:}

\begin{itemize}
{\large \item Bitewing: This technique involves biting down on a special piece of paper so that your dentist can see how well the crowns of your teeth match up. This is commonly used to check for cavities between teeth.
\item Occlusal: Occlusal is a radiograph designed to be placed between the occlusal surfaces of the teeth with the central beam directed at 90 degree or 50 to 60 degree to the plane of the film depending on what is required to be viewed. 
\item Panoramic: A panoramic radiograph is a panoramic scanning dental x-ray of the upper and lower jaw. It shows a two dimensional view of a half circle from ear to ear.
\item Cephalogram: Cephalometric x-rays show a side view of your head, exposing teeth, jaw, and surrounding structures. This technology is considered sfe and often useful or necessary to help professionals evaluate and assist patients.
\item Periapical: This technique focuses on two complete teeth from root to ground.}
\end{itemize}

\subsection{Procedure OF DENTAL X-RAYS}

{\large Step:1: Dental x-rays are taken with you sitting upright in a chair.\\
Step:2: The dental technician will place a lead apron over your chest and wrap a thyroid collar around your neck.\\Step:3: The x-ray sensor or film will be placed in your mouth for the picture.}


\subsection{WHY DENTAL X-RAY IS DONE}

{\large Dental x-rays are important because they give your dentist the whole picture. They help dentist see the condition of your teeth and also the roots, jaw placement, and facial bone composition. They will help your dentist find and treat dental problems before they become too serious or advance.}


\subsection{BENEFITS OF DENTAL X-RAYS}


{\large According to the American dental association, x-rays can show tooth decay, fillings, gum disease, and types of tumors. Dental x-rays can also alert your dentist to changes in your hard and soft tissues. In children, x-rays allow the dentist to see how their teeth and jaw bones are developing.}



\subsection{RISKS OF DENTAL X-RAYS}

{\large While dental x-rays do involve radiation, the exposed levels are so low that they're considered sfe for children and adults. If your dentist uses digital x-rays instead of developing them on film, your risks from radiation exposure are even lower}


\clearpage


\section{REFERENCE}

\begin{itemize}
{\large \item Wikipedia 
\item Healthline.com
\item Mayoclinic.org
\item My.clevelandclinc.org
\item Webmd.com}
\end{itemize}






\end{document}